\documentclass[a4paper,11pt,DIV=11]{scrartcl}
\usepackage[utf8]{inputenc}
\usepackage[T1]{fontenc}
\usepackage[francais]{babel}
\usepackage{setspace}
\usepackage{graphicx}
\usepackage{dirtytalk}


%opening
\title{Rapport de projet: \emph{calcultarice polonaise}}
\author{Lucas Schott}

\begin{document}

\begin{onehalfspace}

    \maketitle
     \\ \\ \\ \\ \\ \\ \\ \\ \\


    \section*{Introduction}


    Le but de ce projet est de réaliser une calculatrice polonaise qui lit des
    expressions arithmétiques au format prefix sur l'entrée standard. Ces expressions arithmétique
    sont à écrire sous forme de chaine de caractères. Ce projet à été réalisé seul en langage C,
    révision C11.
    On peut utiliser le programme de plusieurs manières:

    \begin{itemize}
        \item en envoyant directement les expressions à calculer via un tube:\\
            {\fontfamily{lmtt}\selectfont \$ printf '+ * - 1 2 3 4' | ./pc\\
            1.0000}
        \item de manière interactive, en lancant le programme et en entrant les expressions:\\
            {\fontfamily{lmtt}\selectfont \$ ./pc\\
            + * - 1 2 3 4\\
            1.0000}
            \\
    \end{itemize}


    \section*{Utilisation}


    Pour écrire des expressions, il faut mettre un espace en chaque mots de l'expression, les
    opérateurs, les nombres, les parenthèses ou même {\fontfamily{lmtt}\selectfont \say{\#}}
    pour introduire les commentaires. Le résultat de l'expression écrite est donnée sur la ligne
    en dessous s'il est valide. Si l'expression est invalide, une information sur l'erreur
    rencontrée est donnée, suivit de {\fontfamily{lmtt}\selectfont\mbox{\say{ERROR}}}. Les
    expressions arithmétiques sont composés d'opérateurs et d'opérandes, chaque opérande peut être
    une expression.\\

    Le programme \emph{pc} admet 6 opérateurs binaires: +, -, *, /, min, et max, qui s'utilisent de
    la manière suivante:\\
    {\fontfamily{lmtt}\selectfont <operateur binaire> <operande1> <operande2>\\}

    Il admet 6 opérateurs unaires: exp, ln, abs, floor et ceil qui s'utilisent de cette manière:\\
    {\fontfamily{lmtt}\selectfont <operateur unaire> <operande>\\}

    Le programme admet aussi des parenthèses, les parenthèses permettent uniquement d'augmenter
    l'arité des opérateurs binaires. Quand une parenthèses est ouverte le mot suivant doit être un
    opérateur binaire suivit d'opérandes, l'opérateur s'executera sur toutes les
    opérandes jusqu'à tomber sur une parenthèse fermante. Les parentheses d'utilisent de la
    manière suivante:\\
    {\fontfamily{lmtt}\selectfont ( <operateur binaire> <operande1> \ldots <operandeN> )\\}

    Le programme permet aussi d'affecter une valeur à une variable. La valeur peut être issu d'une
    expression. Le resultat retourné par une affectation est la valeur affectée. Une variable est
    un unique caractère alphabétique minuscule, sa valeur est conservée pour les expression suivantes
    jusqu'à la sortie du programme ou jusqu'a ce qu'on lui réaffecte une nouvelle valeure. Une variable
    déjà initialisée peut être ré-affectée à tout moment. L'affectation fonctionne de la manière
    suivante:\\
    {\fontfamily{lmtt}\selectfont = <variable> <expression>\\}

    Le programme permet aussi d'écrire des commentaires, tout ce qui est marqué à partir du symbole
    \say{\#} et jusqu'à la fin de la ligne est un commmentaire et n'influra en rien sur le résultat
    du calcul.\\
    {\fontfamily{lmtt}\selectfont \# <commentaire>\\
    <expression> \# <commentaire>\\}


    \section*{Implémentation}


    Le programme a été réalisé en langage C, il est composé de deux fichiers \say{.c} et d'un fichier
    \say{.h}: abin.c et abin.h définissent les fonctions sur les arbres binaires, pc.c contient les
    fonctions de lecture des expressions, d'évaluation et le main.\\ 

    \subsection*{Structure de donnée}

    \begin{itemize}
        \item[$\bullet$] un tableau de chaine de caractères \emph{op\_tab} qui contient les chaines
            représentant chaque opérateur. Et \say{\textbackslash{}0} s'il n'y a pas d'opérateur.
        \item[$\bullet$] une structure \emph{noeud} qui contient:
            \begin{itemize}
                \item un entier \emph{op} qui représente un opérateur.
                \item une double \emph{val} qui contient la valeur du noeud si c'est une feuille.
                \item un booleen \emph{unaire} qui vaut true si l'operateur est unaire, false sinon.
                \item deux struct s\_noeud * fg et fd, des pointeurs vers les fils gauche et droit.
            \end{itemize}
        \item[$\bullet$] une structure \emph{resultat} contenant:
            \begin{itemize}
                \item un double nombre qui contient la valeur du résultat.
                \item un booleen valide qui vaut true si le resultat est issu d'une expression
                    ou s'il y a eu une erreur dans l'évaluation de l'expression.
            \end{itemize}
        \item[$\bullet$] une structure \emph{variable} contenant:
            \begin{itemize}
                \item un double valeur qui contient la valeur de la variable.
                \item un bolléen utilise qui vaut true si la variable est initialisée.
            \end{itemize}
        \item[$\bullet$] un tableau de variables \emph{var\_tab} qui contient 26 \emph{structures variable}
            pour stocker les valeurs qui leur sont affectés.
    \end{itemize}
     \\

    \subsection*{Déroulement du programme}

    \begin{itemize}
        \item[$\bullet$] Le programme lit une chaine de caractère dans l'entrée standard avec
            \emph{fges}.
        \item[$\bullet$] Il divise ensuite la chaine obtenue en mots avec \emph{strtok}.
        \item[$\bullet$] Il lit ensuite l'expression mot par mot en enracinant la valeur du mot de
            façon récursive:
            \begin{itemize}
                \item Si le mot est un opérateur binaire, il continuera à lire en enracinant dans
                    le fils gauche jusquà arriver à une feuille. Puis continura à lire en
                    enracinant dans le fils droit.
                \item Si le mot est un opérateur unaire, il continuera à lire en enracinant dans le
                    fils gauche jusqu'à arriver à une feuille..
                \item Si le mot est un nombre, il s'arretera de lire la sous-expression.
                \item si le mot est une parenthese ouvrante, le programme lira le mot suivant qui
                    doit être un operateur binaire ou un nombre:
                    \begin{itemize}
                        \item[-] si c'est un nombre, il va enraciner ce nombre
                        \item[-] si c'est un operateur binaire, il va lire la première opérande qu'il
                            va enraciner à droite, et à gauche il va ré-enraciner l'operateur, il va
                            ainsi lire chaque opérande jusqu'à tomber sur la parenthese fermante.
                    \end{itemize}
                \item si le mot est un \say{=}, il va lire le nom de la variable, puis lire l'expression à
                    affecter à cette variable. La valeur de la variable sera enraciné.
            \end{itemize}
        \item[$\bullet$] Il va pour finir évaluer le resultat de l'expression en parcourant l'arbre en profondeur.
        \item[$\bullet$] Le resultat est affiché sur la sortie standard.
    \end{itemize}


    \section*{Difficultés rencontrées}

    Le programme de base a été relativement simple à réaliser, après un moment de refléxion sur les algorithmes
    et les structures de données à utiliser, l'implémentation a été assez rapide et le programme fonctionnait
    correctement. Les premiers problèmes sont arrivés alors que j'implémentais la gestion des parenthèses
    j'ai mis du temps à trouver tous les cas possible à gérer. Une fois les parenthèses fonctionnelles,
    l'affectation de valeur à des variables a été assez rapide et sans trop de difficultés. Au final j'ai
    pris du temps à réaliser le programme \emph{pc}, mais cela est dù à de nombreuses petites erreurs, je
    n'ai jamais été bloqué très longtemps.


    \section*{Conclusion}

    Ce programme \emph{pc} a été très interressant à réaliser et m'a permis de connaitre de nouveaux outils
    en programmation C tel que \emph{strtok}. Ce projet m'a aussi permis de comprendre et de voir un exemple
    concrêt d'utilisation d'arbres binaire en informatique, et m'a fait prendre concience de l'importance
    d'une structure de donnée claire et efficace. J'ai passé une vingtaine d'heures sur ce projet que j'ai
    trouvé très interressant. Après avoir effectué de nombreux testes et y compris avec valgrind, je pense
    mon programme est opérationnel dans les fonctionnalités que j'ai réalisé.


\end{onehalfspace}

\end{document}
